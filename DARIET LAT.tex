\documentclass[10pt,letterpaper]{article}
\usepackage{cite}
\usepackage{zed-csp,graphicx,color}
\begin{document}
\begin{titlepage}
 \begin{figure}[h]
  \centerline{\small MAKERERE 
  \includegraphics[width=0.1\textwidth]{muklog} UNIVERSITY}
\end{figure}
\centerline{COLLEGE OF COMPUTING AND INFORMATIC SCIENCES}
\paragraph{•}
\centerline{DEPARTMENT OF COMPUTER SCIENCE\\}
\paragraph{•}

\centerline{COURSEWORK THREE: RESEARCH METHODOLOGY(BIT 2207)\\}
\paragraph{•}
\centerline{LECTURER: MR.ERNEST MWEBAZE}
\paragraph{•}
         \centerline{A LITERATURE REVIEW ON GMAIL}
          \author{KAMUKAMA DARIET}
 \paragraph{•}
\centerline{STUDENT NUMBER : 216012298}\
\paragraph{•}
\centerline{REGISTRATION NUMBER:16/U/5346/PS}
\paragraph{•}
%\maketitle
\end{titlepage}

\tableofcontents
\newpage
\pagenumbering{arabic}
\section{introduction.}
Google Optimization Tools (OR-Tools) is a fast and portable software suite for solving combinatorial optimization problems. The suite contains:

A constraint programming solver.
A simple and unified interface to several linear programming and mixed integer programming solvers, including CBC, CLP, GLOP, GLPK, Gurobi, CPLEX, and SCIP.
Graph algorithms (shortest paths, min cost flow, max flow, linear sum assignment).
Algorithms for the Traveling Salesman Problem and Vehicle Routing Problem.
Bin packing and knapsack algorithms.

\section{BODY.}
At launch, Gmail had an initial storage capacity offer of
one gigabyte per user, a significantly higher amount than
competitors offered at the time. Today, the service comes with
15 gigabytes of storage. Users can receive emails up to
50 megabytes in size, including attachments, while they can
send emails up to 25 megabytes. In order to send larger files,
users can insert files from Google Drive into the message.

Gmail has a search oriented
interface and a "conversation
view" similar to an Internet forum. The service is notable
among website developers for its early adoption of Ajax.
Google's mail servers automatically scan emails for multiple
purposes, including to filter spam and malware, and to add
context sensitive

\section{CONCLUSION.}

\bibliographystyle{IEEEtran}
\bibliography{refrences}
\end{document}